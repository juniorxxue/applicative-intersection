\documentclass[a4paper]{article}

\usepackage{geometry}
\geometry{left=2.5cm,right=2.5cm,top=2.5cm,bottom=2.5cm}

% Basics
\usepackage{fixltx2e}
\usepackage{url}
\usepackage{fancyvrb}
\usepackage{mdwlist}  % Miscellaneous list-related commands
\usepackage{xspace}   % Smart spacing
\usepackage{supertabular}

% https://www.nesono.com/?q=book/export/html/347
% Package for inserting TODO statements in nice colorful boxes - so that you
% won’t forget to fix/remove them. To add a todo statement, use something like
% \todo{Find better wording here}.
\usepackage{todonotes}

%% Math
\usepackage{amsmath}
\usepackage{amsthm}
\usepackage{amssymb}
\usepackage{bm}       % Bold symbols in maths mode
\usepackage{wasysym}  % LHD and RHD

% http://tex.stackexchange.com/questions/114151/how-do-i-reference-in-appendix-a-theorem-given-in-the-body
\usepackage{thmtools, thm-restate}

%% Theoretical computer science
\usepackage{stmaryrd}
\usepackage{mathtools}  % For "::=" ( \Coloneqq )

%% Code listings
\usepackage{listings}

%% Font
\usepackage[euler-digits,euler-hat-accent]{eulervm}

% circle (use tikz package)
\newcommand{\ascriptionnode}[1]{
  \tikz{
  \node[shape=circle,draw=black,inner sep=1.5pt,scale=0.3] at (2em, 2em) {#1}; }}


\newcommand{\ascriptioncircle}[2][black,fill=black]{\tikz[baseline=-0.5ex]\draw[#1,radius=#2] (0,0) circle;}

\newcommand{\whitecircle}{\ascriptioncircle[black,fill=white]{2pt}}
\newcommand{\blackcircle}{\ascriptioncircle{2pt}}
\newcommand*\circled[1]{\tikz[baseline=(char.base)]{
    \node[shape=circle,fill,inner sep=1pt] (char) {\textcolor{white}{#1}};}}

%% Ott
\usepackage{ottalt}
\newcommand{\hlmath}[2][gray!40]{%
	\colorbox{#1}{$\displaystyle#2$}}

% Ott includes
\inputott{ott-rules}


\title{Applicative Intersection Types}
\author{Xu Xue}

\begin{document}

\maketitle

\section{Syntax}

\begin{align*}
  &\text{Type} &A, B&::= [[Int]] ~|~ [[Top]] ~|~ [[A-> B]] ~|~ [[ A & B ]]\\
  &\text{Oridinary Type} &O&::= [[Int]] ~|~ [[Top]] ~|~ [[A-> O]]\\
  &\text{Expressions} &e    &::= x ~|~ i ~|~ e:A ~|~ [[e1 e2]] ~|~ [[ \ x : A .e : B]]~|~ [[e_1,,e_2]] \\
  &\text{Partial Values} &p   &::= i~|~ [[ \ x : A .e : B]] \\
  &\text{Values} &v   &::= p:O ~|~ [[v_1,,v_2]]\\
  &\text{Term contexts} &[[T]] &::=  [[ [] ]] ~|~ [[ T, x:A ]] \\
  &\text{Arguments} &[[S]] &::= [[ [] ]] ~|~ A \\
\end{align*}

\section{Subtyping}

\ottdefnsOrdinaryType

\ottdefnsSplitType

\section{Applicative Subtyping}

\ottdefnsUnifiedSubtyping

\section{Typing}

\ottdefnsTyping

\section{Reduction}

\ottdefnsCasting

\ottdefnsParallelApplication

\ottdefnsReduction

% \ottdefnsArgumentalTyping

% \ottdefnsPrincipalTyping

\end{document}
